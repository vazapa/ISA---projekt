\documentclass[a4paper,11pt]{article}
\usepackage[utf8]{inputenc}
\usepackage[T1]{fontenc}
\usepackage{lmodern}
\usepackage[left=14mm,top=23mm,right=14mm,bottom=25mm]{geometry}
\usepackage{amsthm}
\usepackage{amsmath}

\theoremstyle{definition}

\newtheorem{definice}{Definice}

\setcounter{equation}{0}

\begin{document}

\begin{titlepage}
    \begin{center}
        \Huge
        \textsc{Vysoké učení technické v~Brně\\
        \huge Fakulta informačních technologií}\\
        \vspace*{\stretch{0.382}} 
        \LARGE ISA - Síťové aplikace a správa sítí\\
        Aplikace pro získání statistik o síťovém provozu
        \vspace*{\stretch{0.618}}
    \end{center}
    {\Large 2024 \hfill Václav Zapletal (xzaple40)}
\end{titlepage}

\tableofcontents
\newpage

\section{Úvod}
% Tento dokument popisuje implementaci programu \texttt{isa-top}, který slouží k monitorování síťových toků v reálném čase. Program zachycuje síťový provoz a zobrazuje statistiky přenosu dat pro jednotlivá spojení, podobně jako utilita \texttt{top}.
Tento dokument popisuje implementaci programu \texttt{isa-top}, který je využíván k zobrazení statistik o síťovém provozu v terminálu.


\section{Implementace}
Program je rozdělen do několika hlavních komponent:
\subsection{Datové struktury}
Program používá následující klíčové datové struktury:
\begin{itemize}
    \item \texttt{connection\_key\_t} - identifikace spojení (zdrojová/cílová IP, porty, protokol)
    \item \texttt{connection\_stats\_t} - statistiky spojení (počty bajtů, paketů, rychlosti)
    \item Hash tabulka pro ukládání aktivních spojení
\end{itemize}
\subsection{Vlákna}
Z důvodu optimálního a správného výpisu jsou zde využity 2 vlákna vytvořeny pomocí funkce \texttt{pthread\_create}. Tedy jedno vlákno pro výpočet statistik a druhé pro jejich zobrazování pomocí knihovny \texttt{ncurses}.
\subsection{Zachytávání paketů (libpcap)}
Zde je využívana knihovna libpcap, která umožňuje sledovat pakety v síti. Funkce \texttt{create\_pcap\_handle} inicializuje zachytávání na specifikovaném síťovém rozhraní a nastavuje potřebné parametry. Funkce \texttt{packet\_handler}, která je v nekonečné smyčce pomocí \texttt{pcap\_loop}, zpracovává pakety a následně z nich extrahuje informace pro pozdější použití. 

\subsection{Správa spojení (hash tabulka)}
Každé spojení je identifikováno klíčem \texttt{connection\_key\_t}, který obsahuje zdrojovou a cílovou IP adresu, porty a protokol. Tato struktura umožnuje efektivně pracovat se spojeními, které se zde slučují do jedné v případě obousměrné komunikace. Funkce jako \texttt{insert\_or\_update}, \texttt{find} a \texttt{delete} jsou použity pro manipulaci se spojeními v hash tabulce. Když je zachycen nový paket, tato část kódu aktualizuje statistiky příslušného spojení nebo vytvoří nové spojení, pokud dosud neexistuje.
\subsection{Výpočet statistik}
Statistiky zahrnují počet přenesených bajtů a paketů, rychlost přenosu dat a rychlost přenosu paketů. Výpočet statistik se provádí v reálném čase na základě zachycených paketů. Funkce \texttt{update\_speed} aktualizuje rychlosti přenosu dat a paketů pro každé spojení. Kód také zajišťuje, že statistiky jsou pravidelně aktualizovány a stará spojení jsou odstraněna, pokud nejsou aktivní.

\subsection{Uživatelské rozhraní (ncurses)}
Kód zobrazuje seznam aktivních spojení a jejich statistiky v přehledné tabulce. Funkce \texttt{print\_top\_connections} zobrazuje top 10 spojení s nejvyšší přenosovou rychlostí. Uživatelské rozhraní je pravidelně aktualizováno pomocí vlákna \texttt{display\_loop}, aby zobrazovalo aktuální statistiky v reálném čase.

\section{Použití programu}
\subsection{Kompilace}
\begin{verbatim}
make
\end{verbatim}

\subsection{Spuštění}
\begin{verbatim}
./isa-top -i <rozhraní> [-s b/p]
\end{verbatim}

Parametry:
\begin{itemize}
    \item \texttt{-i} - síťové rozhraní
    \item \texttt{-s b} - řazení podle bajtů/s (výchozí)
    \item \texttt{-s p} - řazení podle paketů/s
\end{itemize}

\section{Testování}
Testování probíhalo 2 způsobmi.
\subsection{Ping a Wireshark}
Nejdříve jsem využíval nástroje \texttt{ping}, z několika různých terminálů. Komunikace byla odchytávána pomocí nástroje \texttt{wireshark}.
Toto mi umožnilo určit zdá je zobrazován správný počet bajtů a paketů za sekundu.
\subsection{Iftop}
Pomocí nástroje \texttt{Iftop}, který je velmi podobný \texttt{isa-top} jsem mohl statistiky porovnat a určit zda se rychlosti podobají.

\section{Literatura}
\begin{enumerate}
    \item https://vichargrave.github.io/programming/develop-a-packet-sniffer-with-libpcap/
    \item Dokumentace knihovny libpcap
    \item Dokumentace knihovny ncurses
\end{enumerate}

\end{document}