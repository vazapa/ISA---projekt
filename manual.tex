\documentclass[a4paper,11pt]{article}
\usepackage[utf8]{inputenc}
\usepackage[T1]{fontenc}
\usepackage{lmodern}
\usepackage[left=14mm,top=23mm,right=14mm,bottom=25mm]{geometry}
\usepackage{amsthm}
\usepackage{amsmath}

\theoremstyle{definition}

\newtheorem{definice}{Definice}

\setcounter{equation}{0}

\begin{document}

\begin{titlepage}
    \begin{center}
        \Huge
        \textsc{Vysoké učení technické v~Brně\\
        \huge Fakulta informačních technologií}\\
        \vspace*{\stretch{0.382}} 
        \LARGE ISA - Síťové aplikace a správa sítí\\
        Aplikace pro získání statistik o síťovém provozu
        \vspace*{\stretch{0.618}}
    \end{center}
    {\Large 2024 \hfill Václav Zapletal (xzaple40)}
\end{titlepage}

\tableofcontents
\newpage

\section{Úvod}
% Tento dokument popisuje implementaci programu \texttt{isa-top}, který slouží k monitorování síťových toků v reálném čase. Program zachycuje síťový provoz a zobrazuje statistiky přenosu dat pro jednotlivá spojení, podobně jako utilita \texttt{top}.
Tento dokument popisuje implementaci programu \texttt{isa-top}, který je využíván k zobrazení statistik o síťovém provozu v terminálu.


\section{Návrh aplikace}
Program je rozdělen do několika hlavních komponent:
\begin{itemize}
    \item Zachytávání paketů (libpcap)
    \item Správa spojení (hash tabulka)
    \item Výpočet statistik
    \item Uživatelské rozhraní (ncurses)
\end{itemize}

\section{Implementace}
\subsection{Datové struktury}
Program používá následující klíčové datové struktury:
\begin{itemize}
    \item \texttt{connection\_key\_t} - identifikace spojení (zdrojová/cílová IP, porty, protokol)
    \item \texttt{connection\_stats\_t} - statistiky spojení (počty bajtů, paketů, rychlosti)
    \item Hash tabulka pro ukládání aktivních spojení
\end{itemize}

\subsection{Zpracování paketů}
Implementace zahrnuje:
\begin{itemize}
    \item Parsování IP hlaviček (IPv4/IPv6)
    \item Identifikace protokolů (TCP/UDP/ICMP)
    \item Výpočet přenosových rychlostí
    \item Správu aktivních spojení
\end{itemize}

\subsection{Zajímavé části implementace}
\begin{itemize}
    \item Sloučení obousměrného provozu pro jedno spojení
    \item Formátování rychlostí (K/M/G jednotky)
    \item Řešení localhost spojení
    \item Aktualizace statistik v reálném čase
\end{itemize}

\section{Použití programu}
\subsection{Kompilace}
\begin{verbatim}
make
\end{verbatim}

\subsection{Spuštění}
\begin{verbatim}
./isa-top -i <rozhraní> [-s b/p]
\end{verbatim}

Parametry:
\begin{itemize}
    \item \texttt{-i} - síťové rozhraní
    \item \texttt{-s b} - řazení podle bajtů/s (výchozí)
    \item \texttt{-s p} - řazení podle paketů/s
\end{itemize}

\section{Testování}
\subsection{Testovací prostředí}
\begin{itemize}
    \item Různá síťová rozhraní
    \item IPv4 a IPv6 provoz
    \item TCP, UDP a ICMP protokoly
    \item Vysoké datové toky
\end{itemize}

\subsection{Výsledky testů}
\begin{itemize}
    \item Správné zobrazení IPv4/IPv6 adres
    \item Přesnost měření přenosových rychlostí
    \item Stabilita při dlouhodobém běhu
    \item Správné řazení spojení
\end{itemize}

\section{Literatura}
\begin{enumerate}
    \item STEVENS, W. Richard. TCP/IP Illustrated, Volume 1: The Protocols
    \item Dokumentace knihovny libpcap
    \item Dokumentace knihovny ncurses
\end{enumerate}

\end{document}